\section*{Abstract}
Emotion regulation is the process of consciously altering one's affective state, that is the underlying emotional state such as happiness, confidence, guilt, anger etc. The ability to effectively regulate emotions is necessary for functioning efficiently (for example, adapting across contexts) in everyday life. Individuals are now leveraging the pervasiveness of digital technology to shape their affective states, a process known as digital emotion regulation. Digital Emotion Regulation (DER) is an interdisciplinary field that bridges psychological research into why people regulate their emotions with computing research into how digital technology affects people's emotions. It provides the opportunity to capture the state and context-sensitivity of the emotion regulation process. Understanding digital emotion regulation can thus contribute to a deeper exploration of the essence of emotion regulation as well as the rise of ethical technology design, development, and deployment. 

The desire to regulate one's emotions has been linked to the problematic use of social media and smartphones. Since emotion regulation is such an essential component of overall health, it is necessary to develop interventions that can help people shift their dominant (or modified) coping mechanisms to digital forms in ways that are broadly applicable and manualized. Current emotion regulation tools include mood-based recommendation systems and reminders, which are only temporary and difficult to integrate into daily life. Furthermore, studies show that a combination of online media platforms and apps are used to perform digital emotion regulation. However, these apps or platforms have not been curated for the purpose.

The proposed research focuses on providing comprehensive solutions in the form of a novel framework that recognises the necessity of emotion regulation in online environments and offers on-the-spot assistance for implementation. Using a graph-based framework for analysing online conversations, this work integrates emotion recognition and toxicity detection.
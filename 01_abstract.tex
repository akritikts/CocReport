\section*{Abstract}
Emotion regulation is the process of consciously altering one's affective state, that is the underlying emotional state such as happiness, confidence, guilt, anger etc. The ability to effectively regulate emotions is necessary for functioning efficiently (for example, adapting across contexts) in everyday life. Individuals are now leveraging the pervasiveness of digital technology to shape their affective states, a process known as digital emotion regulation. Digital Emotion Regulation (DER) is an interdisciplinary field that bridges psychological research into why people regulate their emotions with computing research into how digital technology affects people's emotions. It provides the opportunity to capture the state and context-sensitivity of the emotion regulation process. Understanding digital emotion regulation can thus contribute to a deeper exploration of the essence of emotion regulation as well as the rise of ethical technology design, development, and deployment. 

This article offers an overview of digital emotion regulation as well as a synthesis of recent research on digital technology-based emotion regulation interventions. We present our findings from a review of current literature on how different digital applications are used at different stages of the emotion regulation process. We also discuss current gaps in the literature and outline future research directions.
\newpage 
\section{Literature Review}~\label{sec:literature}
Human Computer Interaction (HCI) research has seen an increase in interest in the design and development of technological interventions that can help and support emotion regulation. This renewed interest originates from the realisation and recognition that effective emotion regulation is a vital aspect of mental health \cite{gross2015emotion}, \cite{davidson1998affective}. Over the last decade, there has been a lot of research into technology-enabled mental health support and the development of a diverse array of tools designed to assist with emotion regulation \cite{wadley2020digital}, \cite{smith2022digital}, \cite{slovak2022designing}. The availability and popularity of low-cost wearable sensors and extensively configurable bio-feedback devices fuelled this. Another factor that has contributed to this advancement is psychological research in the field of emotion regulation. It now recognises effective ER as a key factor in personal well-being and is using it as a trans-diagnostic intervention for mental health disorders. This section summarises the recent elements of research and development that have emerged in the field of digital emotion regulation. 

\subsection{Observing Digital Emotion Regulation in Everyday Life}

Research in digital emotion regulation requires observing the entire process of emotion regulation, beginning with recognising the need to regulate, evaluating the context, implementing, monitoring, and modifying strategies. The majority of psychological research in ER is based on static self-reported assessments, experience sampling, surveys, or lab-based tasks collected via questionnaires, interviews, and group discussions, making the state-sensitive task of emotion regulation difficult to capture \cite{wadley2022future}. Participants in this type of data collection are expected to record their interactions with emotion regulation and technology use over a set period of time, and then discuss their use of technology and the emotional reactions that accompany it in an interview.  By analysing the expressible, shareable, consumable, and evaluable emotional affordances, it was discovered that, while these emotional affordances yield emotions and responses, they do not play a role in eudaimonic well-being \cite{tag2021retrospective}, \cite{steinert2022emotions}. These methods allow participants to trace and reflect on significant insights by limiting the amount of detail to be recorded but pose a challenge to advance research in the field.


Recent studies have examined how people combine a variety of social media applications, such as video streaming platforms, discussion forums, online games, and music, to regulate their emotions, thoughts, and behaviours. A common technique known as `mental reset' has been observed, in which individuals employ social media apps to distract themselves from overwhelming emotions \cite{eschler2020emergent}. Several diary studies have been conducted to better understand various aspects of everyday emotion regulation. Wadley et al. investigate the habits of international university students to see if increased access to recorded music via streaming platforms has enabled new ways of regulating emotions \cite{wadley2019use}. According to the findings, students actively and frequently use music streaming services to control their emotional states as a result of the challenges of studying abroad (affective states including stress, boredom, weariness, and loneliness, as well as occasionally joy). The analysis revealed five themes in which students used online music streaming services to help with emotion regulation (both hedonic and instrumental), such as listening to music to boost energy, and focus, and deal with boredom, homesickness, or stress. Smith et al. explored how individuals may have developed a strategic toolkit of emotional resources by combining a variety of applications and devices for purposefully managing emotions in daily life \cite{smith2022digital}. According to the authors, a digitally generated emotional challenge is best met with a digital reaction, such as leaving a comment on a Facebook post to soothe anger. While digital emotion regulation patterns may resemble non-digital emotion regulation patterns, they conclude that digital technologies may subtly alter how these techniques are used by providing a broader range of strategic options that can be easily and effectively implemented. People were also observed using digital tools to enhance non-digital behaviours, such as jogging with music. Individuals' multitasking and passive scrolling habits on social media apps have also been explored in studies revealing how people voluntarily take breaks from social media to mitigate negativity or uphold a sense of equilibrium \cite{lukoff2018makes}, \cite{mark2015focused}, as well as highlighting the practice of interpersonal emotion regulation in discussion forums \cite{smith2022digital}. Similarly, Kelly et al. conducted a diary study of university students' use of technologies to alleviate homesickness and discovered a diverse range of technologies were utilized for this purpose \cite{kelly2021s}. Kou and Gui studied forms of interpersonal regulation within eSport teams by analysing a discussion forum, identifying digitally-mediated forms of regulation such as emoting in a chat channel to motivate teammates and soothe their frustrations \cite{kou2020emotion}. According to Shi et al., smartphone-mediated emotion regulation is effective for achieving desired affective states. However, in lagged analyses, the perceived emotional benefits of smartphone emotion regulation do not emerge, implying that emotional benefits may be transient or reflect self-report biases \cite{shi2023instant}.


Social media usage is categorised as active and passive in research relating it to well-being to explain its conflicting effects on emotional health. While passive social media usage is defined as simply viewing news feeds or ingesting content without interacting with it, active social media usage includes actions like messaging friends, liking or commenting on posts, and publishing status updates. In a recent survey by Hossain et al. to determine which Social Media Multitasking (SMM) behaviours and circumstances contributed to procrastination or recovery, they discovered that both active and passive social media breaks had the potential to be procrastination or recovery activities. The automaticity and situational social media aspects that influence a social media break determine whether it turns into a recovery or procrastination activity. It was concluded that active social media usage had a higher rate of emotion regulation than passive usage, which provided superior support for rest and recovery \cite{hossain2022motivational}. The effect of the pandemic on young people's digital emotion regulation habits has also been studied, and it indicates that while the lockdown made people's emotion regulation practises more uniform, it also increased their reliance on technology and increased their proclivity to use emotion suppression strategies \cite{tag2022impact}.


Computing research in digital emotion regulation draws on psychological research literature by employing Gross's process model for classifying ER strategies and practices, as well as Tamir's motives for identifying the goals of performing ER \cite{slovak2022designing}, \cite{wadley2020digital}. 
Altogether, these studies show that a variety of digital technologies are employed by people for emotion regulation in everyday life. They emphasise the importance of the technologies packed into these devices and the need to boost well-being online, in addition to the evergrowing role that digital technologies play in fostering mental health globally \cite{slovak2022designing}, \cite{wadley2020digital}, \cite{smith2022digital}.

% \begin{figure}[h]
  
%     \centering
%     \includegraphics[width=12cm,height=14cm,keepaspectratio]{ToolsER.png}
%   \caption{Digital interventions for ER\cite{smith2022digital}}
%   \label{fig:ToolsER}
%   \end{figure}
  
  
  
\begin{table}[]
\centering
\caption{Digital interventions for ER}
\label{tab:my-table}
\begin{tabular}{m{6cm}m{10cm}}
\hline
\textbf{ER Strategy Family}                                                                                                             & \textbf{Novel Tools}                                                                                                                                                                                                                                                       \\ \hline
Situation Selection: avoiding a situation that is likely to provoke unpleasant emotions.                                                 & \begin{itemize}
\item Augmented Reality-based ER technologies to portray affect as interaction states to facilitate interpersonal ER \cite{semertzidis2020neo}.
\item Critical Voice in User Interface Design for ER in Online Conversations \cite{kiskola2021applying}, \cite{kou2020emotion}.
\end{itemize}                             \\ \hline
Situation Modification: modifying characteristics of a scenario in order to change its emotional impact.                                 & \begin{itemize}
\item Robot-based artificial commensal companions to facilitate social interactions while eating for people who want or are forced to eat alone \cite{mancini2020room}. Interaction-based technology to enable users up-regulate positive emotions \cite{li2020purpal}
\item Soundscapes for enhancing task performance and mood \cite{newbold2017using}, \cite{yu2018delight}.
\end{itemize}                 \\ \hline
Attentional Deployment: concentrating on or away from elements of a situation that elicit emotion in order to evoke the desired emotion. & \begin{itemize}
\item Customisable virtual reality environments for in-the-moment soothing support for open workplaces \cite{ruvimova2020transport}.
\item Photos with expression and intent to achieve calm/boost desired emotions \cite{chen2016promoting}.
\end{itemize}                                       \\ \hline
Cognitive Change: reevaluating a situation to change its emotional impact.                                                               & \begin{itemize}
\item Personalised breathing pacer to reduce anxiety by inducing explicit ER, primarily used for distraction/reappraisal \cite{miri2020evaluating}.
\item Haptics-based smartwatch intervention for cognitive, physiological, and behavioural changes \cite{costa2019boostmeup}.
\end{itemize}                        \\ \hline
Response Modulation: transforming a current emotional reaction or expression into a more desired one.                                    & \begin{itemize}
\item Innovative toys to help school students improve their ER practices and implicit emotional beliefs through repeated interaction \cite{theofanopoulou2019smart}.
\item Haptics-based guided breathing and pleasant scents to encourage safer driving \cite{paredes2018just}, \cite{dmitrenko2020caroma}.
\end{itemize}                      \\ \hline
\end{tabular}
\end{table}
\raggedbottom

\subsection{Design and Evaluation of Tools for Digital Emotion Regulation}
Emotion sensing has gained popularity in HCI research, implying that relatively prevalent technology can be effectively used for emotion recognition. The various sensors (such as accelerometer, gyroscope, proximity sensor, and microphone) built into devices can be used to detect users' internal and external contexts, including emotions. The above tools allow us to better understand how people utilise technology for appropriate emotion-regulation strategies in their everyday life. A growing number of technology devices capable of both passive and active analysis, such as these, are being used to measure specific emotion-regulation processes. Digital technologies, in particular, allow for the comprehensive analysis of instances where individuals engage in regulation strategies and the means they use to do so or adjust their techniques, as well as how well they work in various circumstances and over time.

This body of research involves the development of interventions that seek to support, develop, or guide emotion regulation skills or help individuals use such skills in challenging situations. The process of emotion regulation occurs in four stages: recognising the need or realising the desire for emotion regulation, selecting an appropriate strategy, implementing it, and then monitoring the regulated state to determine whether additional regulation is required. Technology-enabled interventions have aimed at either supporting a specific ER strategy or boosting emotional awareness during the identification or monitoring stages \cite{slovak2022designing}. 
Table-\ref{tab:my-table} describes some of the recent interventions by classifying them into strategy families based on the process model of ER \cite{wadley2020digital}.
They include experience-based design components which primarily focus on bio-feedback or implicit target responses to nudge users subconsciously toward specific physiological states via haptic interaction, such as imitating heart rate to boost performance by decreasing anxiety, for example guided breathing and pleasant scents for safer driving or haptic-based smartwatch application to cognitive change \cite{smith2022digital}, \cite{paredes2018just}, \cite{dmitrenko2020caroma}, \cite{costa2019boostmeup}. These involve features for both on-the-spot and offline support. Triggers are used to direct or support users during the emotion regulation processes in on-the-spot support systems, which can provide either one-time reminders or ongoing haptic support. Offline support systems use interactivity to create a custom cycle for regulating emotions, such as by visualising a timeline of emotional occurrences to structure users' reflections. Recent advances have also included didactic intervention elements that rely on reminder-based recommender systems, such as suggesting specific ER strategies to users and encouraging them to consider their emotional responses. These works investigate new design possibilities for DER, and by examining how these designs affect users, they offer a new set of directions for this field of study.
\vspace{-0.15cm}
\subsection{Recognising Digital Emotion Regulation}
Methods of exploring DER in social media by combining a smartphone's front camera and touch sensor have also become popular. This data has been used to predict the participants' binary emotional states and can also detect the emotional state while passively using social media. The front camera has been used to measure the degree of joy for a phone session to discover unimodal and polymodal patterns of joy \cite{tag2022emotion}. Researchers contend that by using manipulated images which simulate the realistic image distortion occuring when a person's expressions are captured during phone usage, recognition systems rarely confused expressions of surprise, anger, happiness, and neutrality for other emotions. Attention-based LSTM (Long Short-Term Memory) has also been proposed for a user-independent mobile emotion recognition system that uses smartphone-only or wristband data \cite{yang2021behavioral}.

\begin{table}[h]
\caption{A summary of recent studies aimed at recognising Emotion Regulation in online environments}
\label{tab:recog_table}
\resizebox{\columnwidth}{!}{%
\begin{tabular}{llllll}
\hline
\textbf{Title} &
  \textbf{Purpose} &
  \textbf{\begin{tabular}[c]{@{}l@{}}Data used/ \\ parameters \\ monitored\end{tabular}} &
  \textbf{\begin{tabular}[c]{@{}l@{}}Availability \\ of the dataset\end{tabular}} &
  \textbf{\begin{tabular}[c]{@{}l@{}}Size of the \\ dataset\end{tabular}} &
  \textbf{\begin{tabular}[c]{@{}l@{}}Availability of \\ a reproducible\\ model\end{tabular}} \\ \hline
\begin{tabular}[c]{@{}l@{}}Emotion trajectories\\ in smartphone use: \\ Towards recognizing\\ emotion regulation\\ in the wild \\ \cite{tag2022emotion}\end{tabular} &
  \begin{tabular}[c]{@{}l@{}}Present findings from a \\ field study that measured \\ how joy unfolds during \\ everyday smartphone use.\end{tabular} &
  \begin{tabular}[c]{@{}l@{}}Collected using a \\ customised smartphone\\ application that tracked \\ physical (facial) features\end{tabular} &
  Not available &
  \begin{tabular}[c]{@{}l@{}}The study \\ involved \\ 20 participants,\\ was carried out \\ for 14 days\end{tabular} &
  NA \\ \hline
\begin{tabular}[c]{@{}l@{}}Benchmarking \\ commercial emotion \\ detection systems \\ using realistic \\ distortions of facial\\ image datasets \\ \cite{yang2021benchmarking}\end{tabular} &
  \begin{tabular}[c]{@{}l@{}}Evaluate the performance\\ of commercial emotion\\ detection services\end{tabular} &
  \begin{tabular}[c]{@{}l@{}}Utilised 3 facial \\ expression based \\ datasets\end{tabular} &
  \begin{tabular}[c]{@{}l@{}}Used online \\ datasets \\ \href{https://psycnet.apa.org/doiLanding?doi=10.1037/a0023853
}{(ADFES)}\\ \href{https://www.tandfonline.com/doi/abs/10.1080/02699930903485076
}{(RaFD)}, and\\  \href{https://www.frontiersin.org/articles/10.3389/fpsyg.2014.01516/full
}{(WSEFEP)}\end{tabular} &
  \begin{tabular}[c]{@{}l@{}}838 pictures in\\ total\end{tabular} &
  NA \\ \hline
\begin{tabular}[c]{@{}l@{}}Behavioural and \\ Physiological Signals-\\ Based Deep Multimodal\\ Approach for Mobile \\ Emotion Recognition \\ \cite{yang2021behavioral}\end{tabular} &
  \begin{tabular}[c]{@{}l@{}}Propose a novel attention\\ -based LSTM system that\\  uses a combination of\\ sensors (front camera, \\ microphone, touch panel) \\ from a smartphone and \\ wristband\end{tabular} &
  \begin{tabular}[c]{@{}l@{}}Collected a dataset \\ using a smartphone \\ application where \\ the behavioural and \\ physiological \\ parameters were\\ taken into observation\end{tabular} &
  Not available &
  \begin{tabular}[c]{@{}l@{}}The study \\ involved 45\\ participants\end{tabular} &
  \begin{tabular}[c]{@{}l@{}}Described in\\  the paper\end{tabular} \\ \hline
\begin{tabular}[c]{@{}l@{}}How Do You Feel \\ Online? Exploiting \\ Smartphone Sensors\\ to Detect Transitory\\ Emotions during Social \\ Media Use \\ \cite{ruensuk2020you}\end{tabular} &
  \begin{tabular}[c]{@{}l@{}}Explore the identification \\ of people's emotions \\ when they use social \\ media applications\end{tabular} &
  \begin{tabular}[c]{@{}l@{}}Collected a dataset \\ using various \\ (physical) motion\\ /eye-tracking\\ applications\end{tabular} &
  Not available &
  \begin{tabular}[c]{@{}l@{}}The study\\ involved 20\\ participants\end{tabular} &
  NA \\ \hline
\textbf{\begin{tabular}[c]{@{}l@{}}Encouraging Emotion \\ Regulation in Social \\ Media Conversations \\ through Self-Reflection\end{tabular}} &
  \textbf{\begin{tabular}[c]{@{}l@{}}Identify the contexts\\ that need ER and\\ provide on the spot\\ support for the same\end{tabular}} &
  \textbf{\begin{tabular}[c]{@{}l@{}}Utilised data \\ from Twitter \\ conversations\end{tabular}} &
  \textbf{\begin{tabular}[c]{@{}l@{}}Publicly \\ available\end{tabular}} &
  \textbf{\begin{tabular}[c]{@{}l@{}}Size of\\ the dataset:\\ 180K\end{tabular}} &
  \textbf{\begin{tabular}[c]{@{}l@{}}Described in\\ the paper\end{tabular}} \\ \hline
\end{tabular}%
}
\end{table}

Aside from an individual's desire to regulate their emotions, the ER process involves their surroundings, a situational trigger for emotion, and their attempts to regulate that emotion. It is challenging to measure these variables in a lab setup, so studies have begun investigating ways to recognise them in the wild \cite{wadley2019use}, \cite{smith2022digital}, \cite{martin2021music}. To observe the change of emotions using facial expressions, the front camera of smartphones, touch sensors, eye trackers, and motion sensors have been used, both independently and in combination. A recent study used the device's front camera to measure the degree of joy throughout each phone session and divided people into three groups based on how likely they were to feel joy at the start of a session \cite{tag2022emotion}. Go-getters were users who desired a brief sigh of relief or rejoicing before turning on their phone, targeters were users who desired a uniform and gradually increasing joy as the session progressed, and explorers were users who experienced the polymodal joy that sees a gradual decline before locking the phone. Another study applied image manipulation to simulate the realistic image distortion that occurs when capturing a person's expressions for facial expression-based detection of digital emotion regulation \cite{yang2021benchmarking}, \cite{ruensuk2020you}. According to research, combining modalities improves the accuracy of affect detection in social media tasks.



\subsection{Analysing Emotion Regulation in Social Media Conversations}
Problematic use of social networking sites has been widely associated with maladaptive emotion regulation \cite{yang2020social}, \cite{zsido2021role}, \cite{liu2019adult}. According to observational studies on people's use of social media and digital platforms for emotion regulation, smartphone-mediated emotion regulation is effective in achieving desired affective states \cite{shi2023instant}. However, little research has been conducted to determine when this activity becomes harmful to online well-being. Kiskola et al. use design-based cues to trigger emotion regulation through self-reflection to target the practice of uncivil commenting on social networking sites \cite{kiskola2021applying}. To simulate the process of emotion regulation in social media addicts, Fokker et al. developed a second-order adaptive mental network model. They discovered that if cognitive therapy is successful in improving the regulation by reappraisal, the use of avoidance and suppression can be significantly reduced \cite{fokker2021second}. Another study discourages automatic moderation of online posts and comments while presenting seemingly effective, but problematic design explorations. Following the critical design ideology, they consider how to design conventions in social media could be changed without causing negative behavioural consequences \cite{loizides2020human}. Iqbal et al. look into the issue of verbal aggression online in the context of the BTS Army and find that when the responses are largely rude and irrelevant, the emotion regulation process that should allow the subject to contain their anger fails \cite{iqbal2022emotion}. Therefore, in this work, we propose to notify the user of their emotional impact on the conversation. In order to avoid discarding "what may happen" based on a comment, this work offers "what has happened" as a result of a comment.

\subsection{Identifying the Need for Emotion Regulation in Digital Media}
Uncivil commentary contributes significantly to online toxicity. Content moderation has recently been identified as an intervention to enhance online well-being and minimise toxicity by recognising uncivil remarks based on keywords or underlying emotions \cite{thomas2022s}, \cite{jhaver2021evaluating}. Scalable implementations of machine learning-based moderation approaches have also been explored \cite{gorwa2020algorithmic}, \cite{gillespie2020content}. Algorithmic solutions can be utilised to display users a content analysis (emotion-based) of published comments, which may cause some users to reconsider their posts. For example, the Perspective API, which detects toxic writing as a percentage score from a body of text, when implemented into the comment writing system of the Spanish language news site El Pas, was found to have moderately enhanced the quality of discussions. It is difficult to define the boundaries of incivility or "free expression" online, which is why the goal here is to inform the user of the implications of their comments. This will improve their ability to empathise with other online users and trigger implicit emotion management \cite{walther1993impression}. We argue that the challenge of effective ER in computer-mediated textual communication can be met by presenting factual cues to users, as suggested by Kiskola et al, i.e. supporting emotion regulation through automatic identification of emotional elements, and Slovak et al, who highlight the need for DER interventions that would embed didactic learning into participants' lives \cite{kiskola2021applying}, \cite{slovak2022designing}.  The topic of implicit emotion regulation has recently been discussed in the literature. Unlike explicit emotion regulation, which involves a conscious effort to repress emotional reactions, implicit emotion regulation is effortless and potentially automated \cite{torre2018putting}. As a result, in the context of this study, implicit emotion regulation appears as a potential design approach. Emotion regulation can be enhanced by affect labelling, which simply makes emotionally charged aspects of a conversation more apparent. Kiskola et al. present critical viewpoints on potential solutions by describing and analysing systems that promise to promote emotion regulation through self-reflection \cite{kiskola2021applying}. In this work, we propose to notify the user of their emotional impact on the conversation. Therefore, by providing "what has happened" as a result of a comment, this work suggests an alternative to discarding "what may happen" based on a comment.

\subsection{Supporting on-the-spot Digital Emotion Regulation}
It makes sense that learning explicit emotion regulation calls for knowledge of the various strategies that can be employed in various situations as well as the performance of those strategies. That is, regulating one's emotions involves more than just having enough knowledge \cite{slovak2022designing}. An essential component of ER interventions is the transfer. Allowing people to better cope with daily situations, without ongoing support, is critical for long-term impact. One of the major gaps in existing non-technological interventions is a lack of transfer-enabling support in situ \cite{slovak2016scaffolding}, \cite{antle2018opening}, \cite{antle2019design}. Individual ER strategies are supported, but not equally (situation selection, situation modification, attentional deployment, cognitive change, and response modulation). The majority of interactive components across all delivery mechanisms target response modulation strategies, even though these are theoretically known to be less effective over time. Recent work suggests that digital technologies are already often used by participants for situation selection as part of everyday use. Therefore, in this research, we aim to build a context-based recommender to suggest appropriate emotion regulation strategies when the need for emotion regulation is identified. The goal here is to inform users involved in a conversation about the emotional impact of their actions and to gently nudge them towards a responsible mindset through repeated application of the same.

\subsection{Research Gaps}
\begin{itemize}
    \item Lack of digital solutions for ER: Current emotion regulation tools include mood-based recommendation systems and reminders, which are only temporary and difficult to integrate into daily life. Furthermore, studies show that a combination of online media platforms and apps are used to perform digital emotion regulation. However, these apps or platforms have not been curated for the purpose. Innovative digital solutions that facilitate the transmission and development of emotion regulation practises will not only have a long-term impact, but will also boost social media usage. It is necessary to develop sophisticated tools that allow people to consider, evaluate, and self-diagnose the positive and negative outcomes of their digital activities. There is a lack of techniques for interaction design that use existing or new mechanisms to help users navigate emotional experience trajectories rather than providing recommendations based on their emotional state.
    \item Lack of a prototype for synthesising data, contrast and comparison: The majority of research in the field of digital emotion regulation is based on field studies, ecological momentary assessments (EMA), or physiological sensors combined with facial data, as discussed earlier in the article. Due to their sensitive nature, these datasets cannot be shared, resulting in a lack of data to compare and contrast the results produced, as well as a lack of a common prototype for synthesising emotion regulation. The use of easily accessible data for studying DER will facilitate contrast and comparison.
    \item Lack of a framework to detect/recognise DER online: Although the relevance of situation and context in the process of regulating digital emotions is widely acknowledged, little research in this area has been conducted through field studies and interviews. Understanding a user's online presence and connecting it to their approach to emotion regulation would provide new insights into how technology can help break down and personalise existing cognitive modules into manageable pieces. Empirical studies of people's usage of digital technologies to achieve desired emotional results are required to learn how users can be supported by digital media when undergoing emotion regulation. Because of the complexity of emotional expressions, a large portion of DER research presents qualitative results that are difficult to extend. This opens up opportunities for the use of machine learning and deep learning-based algorithms, which can be used to understand the habits, frequency, and intensity of DER practises.
\end{itemize}


\subsection{Research Questions}

\begin{itemize}
    \item How can we provide on-spot support for emotion regulation in online conversations, extending beyond pattern detection? How can we leverage online environments to promote efficient digital emotion regulation as a form of transferable skill instead of didactic information delivery? How can we break down and tailor effective emotion regulation learning for an individual user?
    \item How can we quantify the expression of various emotions and the differences in their nature of dispersal and virility for social media activities? How can this quantification be used to define toxicity?
    \item How can support for Emotion Regulation be expanded to include a variety of ER strategies rather than just suppression and reappraisal in all contexts? How can information delivery in online ER recommendation environments be customised for users and presented as a transferable skill?
\end{itemize}
\section{A new method for calculating the effect of an induced emotion on a conversation}~\label{sec:approach}
\subsection{Emotion Analysis in Social Media}
Several studies have found that rage-inducing emotions spread faster on social media than positive ones \cite{bacaksizlar2019understanding}, \cite{steinert2022emotions}, \cite{chuai2020anger}, \cite{yi2022depicting}. The amount of rage in a post often determines its virality. This applies to social movements, riots, political posts, and fake news \cite{solovev2022moral}, \cite{mirbabaie2021development}. However, when analysing the emotions in social media posts and performing sentiment analysis to predict the virality or toxicity of an online conversation, this difference in the nature of the diffusion of various emotions is overlooked, and all emotions are weighted equally \cite{yue2019survey}, \cite{nemes2021social}. We believe that taking differences in the nature of emotion diffusion into account will lead to deeper insights into analysing the emotions represented by a post. As a result, we intend to use the empirical analysis of the previous research question, as well as the findings in the literature, to develop a function for assigning weights to the various emotions in a social media conversation.
\subsection{Proposed methodology}
